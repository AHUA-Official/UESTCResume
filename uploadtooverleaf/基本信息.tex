%%原模板参考了https://www.wondercv.com/的模板 说老实话 超级简历真比overleaf好用  就像低代码平台一样   但是overleaf使用latex语法支持更好的扩展性
\documentclass[11pt]{article}
% disable indent globally
\setlength{\parindent}{0pt}
% some general improvements, defines the XeTeX logo
\usepackage{xltxtra}
\usepackage{bookmark}
% use hyperlink for email and url
\usepackage{hyperref}
\hypersetup{hidelinks}
\usepackage{url}
\urlstyle{tt}
\usepackage{multicol}
\usepackage{xcolor}
%%%% 统一一种颜色,偏蓝色,用于section下划线和fontawesome
\definecolor{CVBlue}{RGB}{078,164,239 }
%   RGB值   可以关注微信公众号 阿昆的科研日常   那里有很多配色 我列一下我这个二次元喜欢的配色 (R G  B )   其他可以去看配色方案下面的文件
% 胡桃红    201 071  055  
% 芙芙蓝    078,164,239 
% 宵宫红    205 068  050
% 甘雨蓝    196 216  242
% 流萤绿    059 165  149 
% 电兵蓝   我才不找呢   哼哼哼




%照片   标准一寸白底照片   找照相馆拍   拍的好看一点    192*240 
%  姓名   作为简历开头大标题
%  给出学校标识   
% 给出充足的联系方式 
% 给出博客地址   base   很快入职 + 干半年

\usepackage{calc}
%%%% 利用tikz来定位照片和学校Logo 
\usepackage{graphicx}
\usepackage{tikz}
\usetikzlibrary{calc}
% loading fonts
\usepackage{fontspec}
\usepackage{xeCJK}
\CJKsetecglue{}
%% 取消中文与数字之间间隙
%%%%% 字体需要自己下载安装,注意版权问题。
% 这两种字体应该比较好看,英文Helvetica,中文方正兰亭黑,也是有多种版本,自己试试哪些好看。参考了https://www.wondercv.com/的模板
%%%%% windows系统好像需要先安装字体,之后下面语句就够了
%Main document font
%\setmainfont[
%  BoldFont = HelveticaNeueLTPro-Md.otf ,
%]{HelveticaNeueLTPro-Roman.otf}
%
%\setCJKmainfont[
%BoldFont=Pro_GB18030 DemiBold.otf,
%]{Pro_GB18030.otf}
%%%%% 字体需要自己下载安装,注意版权问题
%%%%% linux系统只需要字体路径就行了,如下
% % Main document font

\setmainfont[
Path = Font/,
  Extension = .otf ,
  BoldFont = HelveticaNeueLTPro-Md.otf ,
]{HelveticaNeueLTPro-Roman.otf}
\setCJKmainfont[
Path = Font/,
  Extension = .otf ,
BoldFont=ProGB18030 DemiBold.otf,
]{ProGB18030.otf}


\usepackage{relsize}
\usepackage{xspace}
\protected\def\Cpp{{C\nolinebreak[4]\hspace{-.05em}\raisebox{.28ex}{\relsize{-1}++}}\xspace} 
% use fontawesome
\usepackage{fontawesome}
%\newfontfamily{\FA}{[FontAwesome.otf]}
\usepackage[
	a4paper,
	left=1.2cm,
	right=1.2cm,
	top=1.5cm,
	bottom=1cm,
	nohead
]{geometry}
\renewcommand{\baselinestretch}{1.2} 
%定义行间距1.2
\usepackage{titlesec}
\usepackage{enumitem}
\setlist{noitemsep}

% removes spacing from items but leaves space around the whole list
%\setlist{nosep} % removes all vertical spacing within and around the list
\setlist[itemize]{topsep=0.25em, leftmargin=*}
\setlist[enumerate]{topsep=0.25em, leftmargin=*}
\titleformat{\section}        
% Customise the \section command 
  {\large\bfseries\raggedright} 
  % Make the \section headers large (\Large), % small capitals (\scshape) and left aligned (\raggedright)
  {}{0em}                      % Can be used to give a prefix to all sections, like 'Section ...'
  {}                           % Can be used to insert code before the heading
  [{\color{CVBlue}\titlerule}]                 % Inserts a horizontal line after the heading
\titlespacing*{\section}{0cm}{*1.6}{*1.2}
\usepackage{siunitx}
\usepackage{amssymb}
%\xeCJKsetup{CJKspace=true}
%\xeCJKDeclareCharClass{CJK}{`0 -> `9}    % 设置 0-9 以 CJK 字体输出
%\normalspacedchars{0,1,2,3,4,5,6,7,8,9} % 0-9 的字符类被还原


\begin{document}

\pagenumbering{gobble} 
%%%% 利用tikz来定位照片,标准一寸照片  192*240     985   
\begin{tikzpicture}[remember picture, overlay]
	\node[anchor = north east] at ($(current page.north east)+(-1cm,-1.2cm)$) {\includegraphics[height=2.8cm]{img/myavator.jpg}};
\end{tikzpicture}%
%%%% 利用tikz来定位学校Logo,这里只在第一页显示,345 × 77 mm 
\begin{tikzpicture}[remember picture, overlay]
	\node[anchor = north west] at ($(current page.north west)+(0.2cm,-0.2cm)$) {\includegraphics[height=10cm]{img/uestc.pdf}};
\end{tikzpicture}%
%%%% 利用tikz来定位页脚栏,电子版简历使用,黑白纸质打印效果可能并不好。这里只在第一页显示,如果需要每页都有,页脚或者background中加入。
\begin{tikzpicture}[remember picture, overlay] 
    \node[anchor = south,fill=CVBlue,draw=none,minimum width=\paperwidth,minimum height=1.5em,align=center,font=\footnotesize,text=white] at ($(current page.south)$) 
    {\faGithubAlt \ \href{}{https://www.yuque.com/ahuaofficial/}\qquad
        \faRssSquare \ \href{github地址}{https://github.com/AHUA-Official} };
        % 围栏填上自己的博客地址 
\end{tikzpicture}

%tikzpicture环境很敏感,注释周围的空格、空行都会引起水平距离或垂直距离的变化,
%
\centerline{\LARGE\bfseries{张磊}}

% 联系信息+个人博客+求职优势(就是稳定干半年+入职快)
% 邮箱最好用你的edu邮箱   但是其实用qq也行  主要是要收到邮件查看响应的速度要快
\centerline{\normalsize{
		\faPhone \ 15282149533 \quad
		\faEnvelopeO \ \href{mailto:mxd@mail.bnu.edu.cn}{3IGAGFJWSGYVAUSGV3@qq.com}}}

\centerline{\normalsize{wx:1528HHFH3 \quad base:成都 \quad github : https://github.com/AHUA-Official }}
% 把你github地址写上去  
\centerline{\normalsize{可连续实习6个月以上 \quad 两周内到岗}}





\end{document}
